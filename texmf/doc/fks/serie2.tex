\documentclass[twoside,fykos]{fksserie}
    
\setcounter{year}{25}
\setcounter{batch}{2}
\deadline{20.~září~2011 18.00}


\begin{document}


\maketitle

%\input{uvod\thebatch.tex}

\problemsheading % section

%\problemtask
%\problemtask

\solutionheading %section
%\resultstab{vysledky}


Tady chci použít vektor\cite{lamport94} $\vect{F}\_G = \sin \pi t \oldpi \phi \theta \rho$, množina $\bb{R}$, bod $\bod{A}$ a $\op{div}$.
Tady je ano\,ne vs. ano ne. Rovnice {na více} řádku
$
\sqrt{\frac{a+b+c}{\eu}} + \sqrt{\frac{a+b+c}{\eu}} =
\sqrt{\frac{a+b+c}{\eu}} + 1 =
1 + \sqrt{\frac{a+b+c}{\eu}} =
\pi + \sqrt{\frac{a+b+c}{\eu}} =
\sqrt{\frac{a+b+c}{\eu}} + \sqrt{\frac{a+b+c}{\eu}} =
\sqrt{\frac{a+b+c}{\eu}} + \eu =
\sqrt{\frac{a+b+c}{\eu}} + \sqrt{\frac{a+b+c}{\eu}} =
\sqrt{\frac{a+b+c}{\eu}} = \sqrt{\frac{a+b+c}{\eu}}
$

Tady následuje víceřádkové odvození\cite{lamport94,lamport95}
\eq[m]{
 f(x) = (x+a)(x+b) \lbl{eq:rce1}\\
 = x^2 + (a+b)x + ab\,.
}

Tady následuje víceřádkové odvození
\[
 f(x) = (x+a)(x+b)\,. \label{eq:rce2}
\]

Tady následuje víceřádkové odvození
\eq{
 f(x) = (x+a)(x+b)\,. \lbl{eq:rce3}
}

Rychlost se měří v \jd{m/s} také se dá napsat \popi{v}{m.s^{-1}} a čas v $\jd{s}$.

  \begin{thebibliography}{9}{\subsection*}

\bibitem{cmejk}
  Cmejkrová, S., Daneš, F., Světlá, J.
  \emph{Jak napsat odborný text.}
  1. vydání. Praha: Leda, 1999.
  256 s. ISBN 80-85927-69-1.

\bibitem{kubr}
  Kubrová, Barbora.
  \emph{WWW prezentace jako nástroj online marketingu.} [online]
  1998, č. 6 [cit. 2008-08-20].
  Dostupný z WWW: \url{http://www.ikaros.cz/Clanek.asp?ID=200203072}. ISSN 1212-5075.

\end{thebibliography}

%\problemsolution
%\problemsolution


Vzpomeneme si na Keplerův třetí zákon, který dává do vztahu oběžné 
doby planet $T$ obíhající centrální slunce s~jejich hlavními 
poloosami $a$. V rovnici \eqref{eq:rce1} bude platit i v~našem případě pro změnu
trajektorie Země
\eq{
	\frac{T_0^2}{a_0^3} = \frac{T_1^2}{a_1^3} \, \ztoho \, a_1 =
	\root 3 \of{ \frac{T_1^2}{T_0^2} } \, a_0 \, ,
}
kde indexy 0 budeme značit počáteční situaci, kdy Země obíhá Slunce
po kružnici s~poloměrem~$a_0 = "1 AU" = "1.50e11 m"$ s~oběžnou dobou~$T_0 =
"365.2 dne"$,
a indexy 1 budou značené veličiny odpovídající 
situaci po změně zemské dráhy (doba oběhu~$T_1 = "372.2 dne"$).

\illfig{obrazek.eps}{}{}
  
Hustota toku sluneční energie ve vzdálenosti~$"1 AU"$ od Slunce se
nazývá {\it sluneční konstanta} a její hodnota je~$S_0 = "1370 W/m^2"$. Ve 
skutečnosti se nejedná o~konstantu, protože v~průběhu roku kolísá
o~cca~$"1.7 \%"$%
\footnote{Nemluvě o~tom, že se i její střední hodnota 
periodicky mění v~průběhu 11letého slunečního cyklu},
ale v~rámci řešení úlohy
ji budeme považovat za konstantní. Hustota toku sluneční energie je 
nepřímo úměrná druhé mocnině vzdálenosti a~ve vzdálenosti~$r$ od 
Slunce ji můžeme vypočítat podle vztahu
\eq{
S_r = \frac{a_0^2}{r^2} S_0 \, .
}
V~přísluní naší nové dráhy je $S\_p = S_0$ z~definice a~v~odsluní
\eq{
S\_a = \frac{a\_p^2}{a\_a^2} S_0 = \frac{1}{\( 2 
\root 3\of{ {T_1^2}/{T_0^2} }-1\)^2} S_0 = "0.95" S_0 = 
"1300 W.m^{-2}" \, .
}

\begin{table}
\centering
\begin{tabular}{llD{.}{,}{-1}}
\toprule
\multicolumn{1}{c}{\popi{l}{mm}} & \multicolumn{1}{c}{\popi{t}{s}} & \multicolumn{1}{c}{\popi{f}{Hz}}\\
\midrule
1 & 2 & 3.0\\
23 & 23 & 9.5\\
4 & 1 & 11.0\\
54 & -7 & -0.012\\
\bottomrule
\end{tabular}
\caption{Nameřené hodnoty}
\end{table}

\begin{thebibliography}{9}{\subsubsection*}

\bibitem{lamport95}
  Leslie Lamport,
  \emph{\LaTeX: A Document Preparation System}.
  Addison Wesley, Massachusetts,
  2nd Edition,
  1994.

\bibitem{lamport98}
  Leslie Lamport,
  \emph{\LaTeX: A Document Preparation System}.
  Addison Wesley, Massachusetts,
  2nd Edition,
  1994.

\end{thebibliography}
Pro odhad teploty budeme předpokládat, že Země je dokonale černé 
těleso a že v~každý okamžik je vyrovnaná bilance zářivého výkonu 
dopadajícího na Zemi a výkonem, která je Zemí vyzařovaná jako 
černým tělesem. Jedná se o~logický předpoklad, protože jinak by Země
nebyla v~tepelné rovnováze a~buď by se neustále ohřívala, nebo 
ochlazovala. Ve skutečnosti má Země tepelnou kapacitu, takže není 
v~tak dokonalé tepelné rovnováze -- ani blízko takové, že by se 
dopadající záření z~jedné strany na Zem okamžitě vyzařovalo všemi 
směry, ale berme to jako první přiblížení. Světelný výkon dopadající
na Zemi, která je dokonalá koule o~poloměru~$R\_Z$, ve vzdálenosti~$r$,
je úměrný průřezu Země a hustotě toku sluneční energie, $
P_r = \pi R\_Z^2 S_r \,.  
$
Výkon, který Země vyzáří na svém celém povrchu, je dle 
Stefanova-Boltzmannova zákona 
\eq{
P = 4 \pi R\_Z^2 M = 4 \pi R\_Z^2 \sigma \tau^4 \, ,
}
kde~$M$ je intenzita vyzařování z~povrchu černého tělesa, $\sigma
= "5.67e-8 W.m^{-2}.K^{-4}"$ Stefanova-Boltzmannova konstanta 
a~$\tau$ je teplota černého tělesa. Vzhledem k~tomu, že se mají 
oba výkony rovnat, dostáváme vzorec pro teplotu Země v~našem přiblížení
\eq{
\tau_r = \root 4\of{\frac{S_r}{4 \sigma}} = \sqrt{\frac{a_0}{2 r}} 
\root 4\of{\frac{S_0}{\sigma}} \,.
}

% \DTLsetseparator{;}
% \DTLloaddb[noheader,keys={poradi,jmeno,skola,a,b,c,d,e,f,g,m,suma,perc,sumasum}]{vysl}{vysledky.csv}
% %\DTLloaddb{vysl}{vysledky.csv}
% 
% \begin{table}[htbp]
%   \caption{}
%   \centering
%   \DTLdisplaydb{vysl}
% \end{table}


%\plotfig[p]{graf.tex}{Graf abc}{}
%\plotfig[p]{graf.tex}{Graf abc $x = \pi$}{}

\subsection{Něco navíc}

Pokud bychom tedy chtěli vyřešit globální oteplování jako ve Futuramě,
kde roboti ovlivnili dráhu Země tak, že rok byl o~týden (robotí pařby)
delší, tak by nás kromě výkyvů teploty v~průběhu roku zejména zajímala
průměrná roční teplota. Respektive i~s~naším relativně primitivním modelem
bychom mohli určit, o~kolik zhruba stupňů by se teplota změnila vůči 
původní teplotě. Za tím účelem můžeme využít druhý Keplerův zákon -- {\it zákon
ploch} -- říkající, že za jednotku času průvodič planety opíše stejnou plochu.%
\footnote{Je to jen jiná formulace zákona zachování momentu hybnosti.}
Pro plošnou rychlost~$w$ pak platí
\eq{
w = \frac{a_1 b_1}{T_1} = \frac{r v_r}{2} ,
}
kde~$b_1$ je vedlejší poloosa elipsy a~$v_r$ je rychlost planety ve 
vzdálenosti~$r$ od Slunce. Pokud bychom chtěli, můžeme vypočítat
i~hodnotu~$w$ s~pomocí vztahu $e = {\sqrt{a_1^2 - b_1^2}}/{a_1}$, 
která pak bude $w = \sqrt{a\_a a\_p} = a_0 
\sqrt{2 \left( {T_1}/{T_0} \right) ^{2/3} -1 }$, ale toto číslo
nebudeme dál potřebovat. Vystačíme~si s~úvahou, že když $w$ je 
konstantní, můžeme vyjádřit oběžnou rychlost jako funkci vzdálenosti~$v_r =
{2w}/{r}$.
Vzhledem k~tomu, že trajektorie Země je elipsa,
můžeme si vybrat souřadnou soustavu, kde Slunce bude v~jejím počátku
a~perihelium bude na ose~$x$ v~kladném směru. Naši elipsu popíšeme v~polárních 
souřadnicích jako 
\eq{
r_\phi = a_1 - (a_1 - a_0) \cos \phi \, ,
}
kde~$\phi$ je úhel měřený právě od perihelu v~kladném smyslu (proti směru 
hodinových ručiček). Jde o aproximaci pro malé excentricity $\epsilon$. 
Obecně můžeme kuželosečky v polárních souřadnicích zapsat ve tvaru 
\eq{
r(\phi) = \frac{a_0}{1-\epsilon\cos\phi}\,,
}
kde $a_0$ je velikost hlavní poloosy a $\epsilon$ je excentricita. Pro 
$\epsilon=0$ jde o kružnici, pro $0<\epsilon<1$ jde o elipsu, pro 
$\epsilon=1$ jde o parabolu a pro $\epsilon>1$ to je jedna větev hyperboly.

Průměrná teplota bude 
\eq{
\bar{\tau} =
\frac{\int_0^{2 \pi} r_\phi \tau_{r_\phi} \, \d\phi }{\int_0^{2 \pi} r_\phi \,
\d\phi } \, .
}
Tyto integrály si můžeme nechat numericky spočítat%
\footnote{Například pomocí stroje na~\url{http://www.wolframalpha.com/}.}
a~vyjde nám, že nemůžeme čekat změnu průměrné roční teploty ani o~celé~$"2 \C"$, 
takže pokud by bylo potřeba Zemi ochladit v~situaci, kdy by bylo všem 
nechutné vedro, ani o~týden delší rok by nestačil.



\seriesheading{Aplikace teorie konformních zobrazení} %section
\subsection[Od elektrostatiky k 2 km/h dynamice]{Od elektrostatiky k $"2 km.h^{-1}"$ dynamice}
text seriálu.

\subsection{Od magentoelektrostatiky ke statice}
text další kapitoly seriálu.

\subsection{Od elektrostatiky k dynamice}
text seriálu.

\resultsheading


\makefooter % adreasa a patička

\end{document}
