%
% @author Michal Koutný <michal@fykos.cz>
%
% @description Auxiliary macros for reading problems database
%              and formatting its output.
%

\RequirePackage{ifthen}
% cooperation with fksfigures input@path change
\ifx\@problemsdir\@undefined
\def\@problemsdir{.}
\fi
\renewcommand\@problemsdir{.}
\newcommand\problemsdir[1]{\renewcommand\@problemsdir{#1}}

% problems
\newcommand\problemsfileprefix{} % possibility to override standard filename (for inter-year import)
\newcommand\suppressprobfig{\renewcommand\probfig[1]{\renewcommand\@probfig{N/A}}} % when rendering figure manually
\newcommand\revertprobfig{\renewcommand\probfig[1]{\renewcommand\@probfig{##1}}} % when rendering figure manually

\newcommand\@tmpfilename{}
\newcommand\@LoadProblem[3]{ % batch number, problem number, expansion code
  \renewcommand\@tmpfilename{\@problemsdir/\problemsfileprefix problem#1-#2.tex}
  \probfig{N/A}\probsolvers{N/A}\probavg{N/A}
  \IfFileExists{\@tmpfilename}{
  \input{\@tmpfilename}%
  \ClassInfo{fksserie}{Loaded \@tmpfilename}
  #3
  }{\ClassWarning{fksserie}{Problem file \@tmpfilename not found.}}
}

\newcommand\problemtask{%
  \tasktrue\solutionfalse\solutionsinglefalse%
  \stepcounter{problem}%
  \@LoadProblem{\thebatch}{\theproblem}{%
   \problem[\metaprobletter{problem}: \@probname]{Úloha \problemnum{batch}{problem} \,\ldots{} \@probname{} \hfill {\normalfont\normalsize\problempoints}}%
   \nopagebreak\ifthenelse{\equal{\@probfig}{N/A}}{}{\@probfig}\nopagebreak%
   \@probtask%
  }
}

\newcommand\problemsolution{%
  \taskfalse\solutiontrue\solutionsinglefalse%
  \stepcounter{problem}%
  \@LoadProblem{\thesolvedbatch}{\theproblem}{%
  \pr@blemsolution
}}

\newcommand\myprint[1]{pdfauthor=#1}
\def\expandargument#1#2{%
    \edef\temp{#2}%
    \expandafter#1\expandafter{\temp}%
}

\newcommand\problemsolutionsingle{%
  \taskfalse\solutiontrue\solutionsingletrue%
  \setcounter{solvedbatch}{\@probbatch}
  \setcounter{problem}{\@probno}
  \hypersetup{
    pdftitle={\met@shortname, \Roman{year}.\Roman{solvedbatch}.\arabic{problem} \@probname},
    pdfauthor={\@probsolauthors},
  }

  \pr@blemsolution
}

\newcommand\pr@blemsolution{%  
   \problem[\metaprobletter{problem}: \@probname]%
    {Úloha \problemnum{solvedbatch}{problem} \,\ldots{} \@probname{} \hfill
      {\normalfont\normalsize\problemstats}}%
   \nopagebreak\ifthenelse{\equal{\@probfig}{N/A}}{}{\@probfig}\nopagebreak%   
   \textsl{\@probtask}
   \signed{\textit{\@proborigin}}

   \medskip
   
   \noindent \@probsolution\par
   \nopagebreak\vspace{6pt}\nopagebreak\expandafter\signature\@probsolauthors\par % email on the end solution
}

% problem info fields macros
\newcommand\problemnum[2]{\Roman{#1}.\metaprobletter{#2}}

\newcommand\problemstats{\problempoints;
\ifthenelse{\equal\@probavg{N/A}}{(chybí statistiky)\PackageWarning{fks problems}{Missing stats for problem \Roman{solvedbatch}.\theproblem.}}{%
průměr~\@probavg;
\plural{\@probsolvers}{řešil}{řešili}{řešilo}{}~\@probsolvers{}~\plural{\@probsolvers}{student}{studenti}{studentů}}}

\newcommand\problempoints{\@probpoints~\plural{\@probpoints}{bod}{body}{bodů}}

% ifs for customizing problems layout
\newif\iftask
\newif\ifsolution
\newif\ifsolutionsingle % solutionsingle implies solution


% problem attributes
\newcommand\@probbatch{N/A}
\newcommand\@probno{N/A}
\newcommand\@probname{N/A}
\newcommand\@proborigin{N/A}
\newcommand\@probpoints{N/A}
\newcommand\@probsolauthors{N/A}
\newcommand\@probauthors{N/A}
\newcommand\@probsolvers{N/A}
\newcommand\@probavg{N/A}
\newcommand\@probtask{N/A}
\newcommand\@probsolution{N/A}
\newcommand\@probfig{N/A}

\newcommand\probbatch[1]{\renewcommand\@probbatch{#1}}
\newcommand\probno[1]{\renewcommand\@probno{#1}}
\newcommand\probname[1]{\renewcommand\@probname{#1}}
\newcommand\proborigin[1]{\renewcommand\@proborigin{#1}}
\newcommand\probpoints[1]{\renewcommand\@probpoints{#1}}
\newcommand\probsolauthors[1]{\renewcommand\@probsolauthors{{#1}}} % due to 
%foreach expansion
\newcommand\probauthors[1]{\renewcommand\@probauthors{{#1}}}
\newcommand\probsolvers[1]{\renewcommand\@probsolvers{#1}}
\newcommand\probavg[1]{\renewcommand\@probavg{#1}}
\newcommand\probtask[1]{\renewcommand\@probtask{#1}}
\newcommand\probsolution[1]{\renewcommand\@probsolution{#1}}
\newcommand\probfig[1]{\renewcommand\@probfig{#1}}
