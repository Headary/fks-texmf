\documentclass[twoside]{fksserie}
    
\setcounter{year}{25}
\setcounter{batch}{2}
\deadline{20.~září~2011 18.00}



\begin{document}


\maketitle

%\input{uvod\thebatch.tex}

\problemsheading % section

%\problemtask
%\problemtask

\solutionheading %section

Tady je ano\,ne vs. ano ne. Rovnice na více řádku
$
\sqrt{\frac{a+b+c}{\eu}} + \sqrt{\frac{a+b+c}{\eu}} =
\sqrt{\frac{a+b+c}{\eu}} + 1 =
1 + \sqrt{\frac{a+b+c}{\eu}} =
\pi + \sqrt{\frac{a+b+c}{\eu}} =
\sqrt{\frac{a+b+c}{\eu}} + \sqrt{\frac{a+b+c}{\eu}} =
\sqrt{\frac{a+b+c}{\eu}} + \eu =
\sqrt{\frac{a+b+c}{\eu}} + \sqrt{\frac{a+b+c}{\eu}} =
\sqrt{\frac{a+b+c}{\eu}} = \sqrt{\frac{a+b+c}{\eu}}
$

A následuje nová $"18.9"$ rovnice na samostatné řádku
$$
  2*3 = 0.6\E1 = "18.6 km/h" = "58.1 m.s^{-1}"
$$

Rychlost se měří v \jd{m/s} také se dá napsat \popi{v}{m.s^{-1}} a čas v $\jd{s}$.

%\problemsolution
%\problemsolution

% \begin{comment}
% Vzpomeneme si na Keplerův třetí zákon, který dává do vztahu oběžné 
% doby planet $T$ obíhající centrální slunce s~jejich hlavními 
% poloosami $a$. Stejně bude platit i v~našem případě pro změnu
% trajektorie Země
% $$
% 	\frac{T_0^2}{a_0^3} = \frac{T_1^2}{a_1^3} \, \ztoho \, a_1 =
% 	\root 3 \of{ \frac{T_1^2}{T_0^2} } \, a_0 \, ,
% $$
% kde indexy 0 budeme značit počáteční situaci, kdy Země obíhá Slunce
% po kružnici s~poloměrem~$a_0 = "1 AU" = "1,50e11 m"$ s~oběžnou dobou~$T_0 =
% "365,2 dne"$,
% a indexy 1 budou značené veličiny odpovídající 
% situaci po změně zemské dráhy (doba oběhu~$T_1 = "372,2 dne"$).
% 
% Vzhledem k~tomu, že přechod na eliptickou%
% \footnote{Případně více eliptickou, pokud bychom se rovnou
% rozhodli uvažovat i to, že původní dráha Země je ve skutečnosti
% eliptická s~excentricitou~$e = 0,0167$.}
% dráhu se uskutečnil rychle a ve směru pohybu Země, což znamená, že
% přísluní (perihelium) nové dráhy bude ve vzdálenosti $a\_p = a_0$ od 
% Slunce a~odsluní (afelium) bude ve vzdálenosti~$ a\_a = 2 a_1 - a_0 = \( 2 \root 3\of{ {T_1^2}/{T_0^2} } - 1 \) a_0 \approx  "1,025 AU"$.
% Už z~tohoto výsledku je vidět, že dramatické změny teplot
% v~průběhu roku nebudou nastávat, protože excentricita této dráhy je
% pouze~$e_1 = {(a\_a - a\_p)}/{(a\_a + a\_p)} = 1 - 
% \root 3\of{{T_0^2}/{T_1^2}} = 0,0126$, což je menší excentricita, 
% než má Země ve~skutečnosti. Pokud bychom ale uvažovali eliptickou 
% dráhu, záleželo by na tom, kdy v~průběhu roku dojde ke změně 
% dráhy. Excentricita by se pak mohla i zmenšit, střední vzdálenost
% Země-Slunce by vzrostla v~každém případě tak, aby se velká poloosa
% zvětšila z~$a_0$ na~$a_1$.
% 
% Hustota toku sluneční energie ve vzdálenosti~$"1 AU"$ od Slunce se
% nazývá {\it sluneční konstanta} a její hodnota je~$S_0 = "1370 W/m^2"$. Ve 
% skutečnosti se nejedná o~konstantu, protože v~průběhu roku kolísá
% o~cca~$"1,7 \%"$%
% \footnote{Nemluvě o~tom, že se i její střední hodnota 
% periodicky mění v~průběhu 11letého slunečního cyklu.},
% ale v~rámci řešení úlohy
% ji budeme považovat za konstantní. Hustota toku sluneční energie je 
% nepřímo úměrná druhé mocnině vzdálenosti a~ve vzdálenosti~$r$ od 
% Slunce ji můžeme vypočítat podle vztahu
% $$
% S_r = \frac{a_0^2}{r^2} S_0 \, .
% $$
% V~přísluní naší nové dráhy je $S\_p = S_0$ z~definice a~v~odsluní
% $$
% S\_a = \frac{a\_p^2}{a\_a^2} S_0 = \frac{1}{\( 2 
% \root 3\of{ {T_1^2}/{T_0^2} }-1\)^2} S_0 = 0,95 S_0 = 
% "1300 W.m^{-2}" \, .
% $$
% 
% Pro odhad teploty budeme předpokládat, že Země je dokonale černé 
% těleso a že v~každý okamžik je vyrovnaná bilance zářivého výkonu 
% dopadajícího na Zemi a výkonem, která je Zemí vyzařovaná jako 
% černým tělesem. Jedná se o~logický předpoklad, protože jinak by Země
% nebyla v~tepelné rovnováze a~buď by se neustále ohřívala, nebo 
% ochlazovala. Ve skutečnosti má Země tepelnou kapacitu, takže není 
% v~tak dokonalé tepelné rovnováze -- ani blízko takové, že by se 
% dopadající záření z~jedné strany na Zem okamžitě vyzařovalo všemi 
% směry, ale berme to jako první přiblížení. Světelný výkon dopadající
% na Zemi, která je dokonalá koule o~poloměru~$R\_Z$, ve vzdálenosti~$r$,
% je úměrný průřezu Země a hustotě toku sluneční energie, $
% P_r = \pi R\_Z^2 S_r \,.  
% $
% Výkon, který Země vyzáří na svém celém povrchu, je dle 
% Stefanova-Boltzmannova zákona 
% $$
% P = 4 \pi R\_Z^2 M = 4 \pi R\_Z^2 \sigma \tau^4 \, ,
% $$
% kde~$M$ je intenzita vyzařování z~povrchu černého tělesa, $\sigma
% = "5,67e-8 W.m^{-2}.K^{-4}"$ Stefanova-Boltzmannova konstanta 
% a~$\tau$ je teplota černého tělesa. Vzhledem k~tomu, že se mají 
% oba výkony rovnat, dostáváme vzorec pro teplotu Země v~našem přiblížení
% $$
% \tau_r = \root 4\of{\frac{S_r}{4 \sigma}} = \sqrt{\frac{a_0}{2 r}} 
% \root 4\of{\frac{S_0}{\sigma}} \,.
% $$
% Teplota v~perihelu pak vyjde $\tau\_p \approx "6 \C"$ a v~afelu $\tau\_a \approx
% "2 \C"$. Teplota v~perihelu by teoreticky podle našich předpokladů
% měla odpovídat střední teplotě na Zemi v~průběhu roku, která se udává
% jako~$"14 \C"$. Což na první pohled úplně nesedí, ale vzhledem k~počtu
% zanedbání, kterých jsme se dopustili, je to poměrně dobrá shoda. Další 
% vlivy, které by se pro správné určení teploty měly započítat, jsou 
% například to, že ve skutečnosti spektrum Země při vyzařování nebude 
% ideálně odpovídat vyzařování černého tělesu, ale mělo by určitou 
% specifickou vyzařovací charakteristiku, navíc i tato celková 
% charakteristika by byla jenom přiblížením, protože Země není jenom 
% z~jedné chemické látky, ale jinak bude vyzařovat pevnina a jinak 
% oceány. Toto by vedlo spíš ke snížení očekávané teploty Země. Vliv na
% teplotu Země má také to, že má horké jádro -- částečně obsahující 
% tepelnou energii od doby vzniku Země pocházející z~gravitační
% potenciální energie a dále v~jádru dochází k~rozpadu radioaktivních 
% prvků, což také zvyšuje teplotu Země.
% Další věcí je přítomnost atmosféry, která díky skleníkovým
% plynům zvyšuje teplotu zemského povrchu.
% 
% \section{Něco navíc}
% 
% Pokud bychom tedy chtěli vyřešit globální oteplování jako ve Futuramě,
% kde roboti ovlivnili dráhu Země tak, že rok byl o~týden (robotí pařby)
% delší, tak by nás kromě výkyvů teploty v~průběhu roku zejména zajímala
% průměrná roční teplota. Respektive i~s~naším relativně primitivním modelem
% bychom mohli určit, o~kolik zhruba stupňů by se teplota změnila vůči 
% původní teplotě. Za tím účelem můžeme využít druhý Keplerův zákon -- {\it zákon
% ploch} -- říkající, že za jednotku času průvodič planety opíše stejnou plochu.%
% \footnote{Je to jen jiná formulace zákona zachování momentu hybnosti.}
% Pro plošnou rychlost~$w$ pak platí
% $$
% w = \frac{a_1 b_1}{T_1} = \frac{r v_r}{2} ,
% $$
% kde~$b_1$ je vedlejší poloosa elipsy a~$v_r$ je rychlost planety ve 
% vzdálenosti~$r$ od Slunce. Pokud bychom chtěli, můžeme vypočítat
% i~hodnotu~$w$ s~pomocí vztahu $e = {\sqrt{a_1^2 - b_1^2}}/{a_1}$, 
% která pak bude $w = \sqrt{a\_a a\_p} = a_0 
% \sqrt{2 \left( {T_1}/{T_0} \right) ^{2/3} -1 }$, ale toto číslo
% nebudeme dál potřebovat. Vystačíme~si s~úvahou, že když $w$ je 
% konstantní, můžeme vyjádřit oběžnou rychlost jako funkci vzdálenosti~$v_r =
% {2w}/{r}$.
% Vzhledem k~tomu, že trajektorie Země je elipsa,
% můžeme si vybrat souřadnou soustavu, kde Slunce bude v~jejím počátku
% a~perihelium bude na ose~$x$ v~kladném směru. Naši elipsu popíšeme v~polárních 
% souřadnicích jako 
% $$
% r_\phi = a_1 - (a_1 - a_0) \cos \phi \, ,
% $$
% kde~$\phi$ je úhel měřený právě od perihelu v~kladném smyslu (proti směru 
% hodinových ručiček). Jde o aproximaci pro malé excentricity $\epsilon$. 
% Obecně můžeme kuželosečky v polárních souřadnicích zapsat ve tvaru 
% $$
% r(\phi) = \frac{a_0}{1-\epsilon\cos\phi}\,,
% $$
% kde $a_0$ je velikost hlavní poloosy a $\epsilon$ je excentricita. Pro 
% $\epsilon=0$ jde o kružnici, pro $0<\epsilon<1$ jde o elipsu, pro 
% $\epsilon=1$ jde o parabolu a pro $\epsilon>1$ to je jedna větev hyperboly.
% 
% Pokud jste se ještě s~polárními souřadnicemi nesetkali,
% tak místo souřadnice~$x$ a~$y$ máme souřadnice~$r = \sqrt{x^2 + y^2}$ 
% určující vzdálenost od počátku a~$\phi$, což je právě zmíněný úhel, pro
% který platí $\phi = \tg {y}/{x}$.
% 
% Poslední úvaha se týká toho, že  teplotu bychom chtěli \uv{vystředovat} tak, že
% bychom si rozdělili dráhu 
% Země v~průběhu roku na malé kousíčky, kdy má skoro stejnou teplotu
% určenou naším modelem, a~teplotu vynásobili časem, za který Země příslušný
% kousíček dráhy urazila. Všechny tyto vynásobené kousky bychom pak sečetli
% a~vydělili dobou oběhu. Vlastně bychom spočítali vážený průměr teploty. 
% Čas, který Zemi potrvá, než urazí nějakou dráhu, je 
% nepřímo úměrný její rychlosti. Rychlost je zase v~našem případě nepřímo
% úměrná vzdálenosti od Slunce, takže čas je úměrný vzdálenosti. Takže
% můžeme jako váhovou funkci použít vzdálenost a ne přímo čas. Také
% bude lepší, když kousíčky, ve kterých považujeme rychlost Země za konstantní,
% půjdou k~nekonečně krátkým dobám -- tzn. přejdeme k~integrování. 
% Průměrná teplota bude 
% $$
% \bar{\tau} =
% \frac{\int_0^{2 \pi} r_\phi \tau_{r_\phi} \, \d\phi }{\int_0^{2 \pi} r_\phi \,
% \d\phi } \, .
% $$
% Tyto integrály si můžeme nechat numericky spočítat%
% \footnote{Například pomocí stroje na~{\eighttt http://www.wolframalpha.com/}.}
% a~vyjde nám, že nemůžeme čekat změnu průměrné roční teploty ani o~celé~$"2 \C"$, 
% takže pokud by bylo potřeba Zemi ochladit v~situaci, kdy by bylo všem 
% nechutné vedro, ani o~týden delší rok by nestačil.
% \end{comment}

\seriesheading{Aplikace teorie konformních zobrazení} %section
\subsection{Od elektrostatiky k dynamice}
text seriálu.

\subsection{Od magentoelektrostatiky ke statice}
text další kapitoly seriálu.

\subsection{Od elektrostatiky k dynamice}
text seriálu.

\subsection{Od magentoelektrostatiky ke statice}
text další kapitoly seriálu.

\subsection{Od elektrostatiky k dynamice}
text seriálu.

\subsection{Od magentoelektrostatiky ke statice}
text další kapitoly seriálu.



\makefooter % adreasa a patička

\end{document}
