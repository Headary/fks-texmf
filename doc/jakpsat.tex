\documentclass[fykos]{fksbase} % dle názvu semináře
\setcounter{year}{25}
\fancyhead{}
\fancyhead[L]{\bfseries\small \metavar{headername}}
\fancyhead[R]{\bfseries\small Jak psát vzorové řešení}
\usepackage{multicol}

\begin{document}
\section{Jak psát autorská řešení}

Řešení pište bez chyb! Kdo tam bude mít chyby, bude zmrskán!

Všechny texty týkající se FYKOSu jsou psány v typografickém systému \TeX
(využívajíc {\LaTeX}ových maker). Řešení úloh pište v libovolném
textovém editoru do souboru úlohy v kódování UTF-8. Výsledný soubor buď uložte
do repozitáře, nebo pošlete e-mailem TeXaři.

Text pište česky s háčky a čárkami v první osobě množného čísla, asi po 60~znacích
(po~logických celcích) se snažte odřádkovávat. Více mezer a (jednotlivé) odřádkování
nemá vliv na vizuální podobu výsledného textu. Nový odstavec vytvoříte prázdným
řádkem. Velká většina příkazů
se v TeXu uvozuje zpětným lomítkem (\verb|\|). Například text do uvozovek dáte
pomocí~\verb|\uv{}|.

\subsection{Matematické vzorce}
\begin{compactitem}
    \item Inline matematiku píšeme mezi dolary (např.~\verb|rychlost~$v_1 = w_0$|).
    \item Jednořádkovou blokovou matematiku píšeme mezi \verb|\eq{}| (ekvivalent \verb|$$|)
    a víceřádkovou mezi \verb|\eq[m]{}| (obdoba \verb|align|) a před znaky, na než chceme
    zarovnat dáme \verb|&|.
    \item Blokovou matematiku odsazujeme čtyřmi mezerami pro lepší čitelnost.
    \item Odkazy děláme pomocí \verb|\lbl| na konkrétní řádek.
    \item Vzorec se chová jako část textu (čárku, tečku na konci řádku oddělujeme mezerou~\verb|\,|).
    \item Ve vzorcích používejte normálně znaménka plus, mínus; znaménko krát není vhodné psát.
    \item Horní a dolní index vytvoříte pomocí stříšky (\verb|^|) nebo podtržítka~(\verb|_|).
    \item Pokud je v indexu více znaků, dejte je do složených závorek. Index, který
    vznikl z nějakého slova, se píše stojatý – před podtržítko dejte zpětné lomítko~(\verb|\_|).
    \item Čárku nad veličinou pište pomocí anglického apostrofu (\verb|'|).
\end{compactitem}

Snažte se o to, aby zdrojový kód byl co nejpřehlednější. Pokud nevíte, jaká je
přesná syntaxe určitého příkazu,
Google a manuál {\LaTeX}u vám poradí. Poznámky, které se nemají objevit ve
výsledném textu, pište na řádek začínající \verb|%|.

\subsubsection{Příklady příkazů}

Hodit se vám budou následující příkazy:
\begin{multicols}{2}
\begin{compactitem}
    \item \verb|\frac{}{}| zlomek, do závorek napíšete čitatele a jmenovatele,
    \item \verb|\cdot| tečka mezi dvěma zlomky,
    \item \verb|\sqrt{}| odmocnina,
    \item \verb|\alpha| malá alfa (ostatní řecká písmena obdobně),
    \item \verb|\Alpha| velká alfa,
    \item \verb|\int| integrál,
    \item \verb|\sum| součet,
    \item \verb|\sin{}| sinus,
    \item \verb|\cos{}| kosinus,
    \item \verb|\dot{}| tečka nad znakem (např. časová derivace),
    \item \verb|\ddot{}| dvě tečky (druhá časová derivace),
    \item \verb|\%| procento,
    \item \verb|\dots| tři tečky,
    \item \verb|\cdots| centrované tři tečky,
    \item \verb|\footnote{}| poznámka pod čarou.
\end{compactitem} 
\end{multicols}

\newpage
Další příkazy se týkají pouze FYKOSu, jsou definovány ve FYKOSích makrech
(balíček~\verb|fkssugar|).\nopagebreak
\begin{multicols}{2}
\begin{compactitem}
    \item \verb|\d| diferenciál,
    \item \verb|\der{y}{x}| první derivace $y$ podle $x$,
    \item \verb|\dder{}{}| druhá derivace,
    \item \verb|\pder{}{}| parciální derivace,
    \item \verb|\vect{}| vektor vysázený tučně,
    \item \verb|\bb{}| množiny (např. $\bb{R}$),
    \item \verb|\eu| Eulerovo číslo,
    \item \verb|\tg{}| tangens,
    \item \verb|\(| \verb|\)| levá a pravá závorka,
    \item \verb|\ztoho| implikace,
    \item \verb|\const| konstanta,
    \item \verb|\dg| stupeň,
    \item \verb|\im| imaginární jednotka,
    \item \verb|\bod{A}| geometrický bod~$\bod{A}$,
    \item \verb|\C| stupně Celsia,
    \item \verb|\pi| kontanta $\approx "3.14"$,
    \item \verb|\oldpi| úhel, rovina, paralaxa.
\end{compactitem}
\end{multicols}

\subsection{Čísla a jednotky}

Chceme-li napsat desetinné číslo, příp. číslo s jednotkami, použijeme uvozovkového
pomocníka (je citlivý na mezeru před jednotkou). Čísla píšeme s desetinnou
tečkou(!) (snadno pak naformátujeme i výstup z Gnuplotu).

\begin{verbatim}
Teplota $"21.3 \C"$, úhel je $"123\dg", $\sigma = "5.67e-8 J.s^{-1}.m^{-2}.K^{-4}"$
$"(0.7687\pm 0.0003) arcsec"$, vzestup o $"13.2 \%"$.
\end{verbatim}


\subsection{Obrázky}

Obrázky jsou rozděleny do tří kategorií: grafy, drobné ilustrační, datové/velké
ilustrační.
\smallskip

\noindent\verb|\plotfig{|{\tt \itshape cesta k souboru .tex}\verb|}{|{\tt \itshape popiska}\verb|}{|{\tt \itshape referenční ID}\verb|} % Gnuplot term eps latex|\\
\verb|\fullfig{|{\tt \itshape cesta k souboru}\verb|}{|{\tt \itshape popiska}\verb|}{|{\tt \itshape referenční ID}\verb|}|\\
\verb|\illfig{|{\tt \itshape cesta k souboru}\verb|}{|{\tt \itshape popiska}\verb|}{|{\tt \itshape referenční ID}\verb|}{|{\tt \itshape výška v řádcích}\verb|}|
\smallskip

\emph{Poznámka}\quad Jediným povinným argumentem je cesta k souboru
(implicitně se prohledávají adresáře \verb|graphics|).
Ostatní argumenty lze též zadat jako prázdné tokenem \verb|{}|.

Obrázky (i zdrojáky) k úlohám se ukládají do adresáře \verb|problems/graphics|.
Ostatní obrázky (seriál, úvodníček) se ukládají k dané sérii
(\verb|batchB/graphics|).

\subsection{Ostatní}

Seznam vytvoříte pomocí {\LaTeX}ového prostředí~\verb|compactitem| nebo~\verb|compactenum|,
každou položku označte~\verb|\item|.

Na webové odkazy použijte makro \verb|\url{|{\tt \itshape adresa}\verb|}|. Odkazy do textu vložíte pomocí
maker~\verb|\ref{|{\tt \itshape referenční ID}\verb|}| (obrázky, tabulky) a~\verb|\eqref{|{\tt \itshape referenční ID}\verb|}| (pro rovnice).

Text pište v první osobě množného čísla, na konci svého řešení můžete uvést
hlavní chyby, kterých se řešitelé dopouštěli.
\end{document}
